\documentclass[a4paper,12pt]{article}
\usepackage[utf8]{inputenc}
\usepackage[T1]{fontenc}
\usepackage{imfellEnglish}
\usepackage[spanish]{babel}
\usepackage{amsmath}
\usepackage{tikz}
\usepackage{geometry}
\usepackage{siunitx}
\usepackage{xcolor}
\usepackage[most]{tcolorbox}
\usepackage[italic, basic]{mathastext}
\usepackage{caption}
\usetikzlibrary{babel, tikzmark}

\geometry{margin=1.5cm}

\begin{document}
\itshape

\begin{tcolorbox}[
    enhanced,
    colback=white,
    colframe=black,
    arc=0pt,
    outer arc=0pt,
    boxrule=0.5mm
]

Una barra de longitud $L$ se encuentra en dirección perpendicular a una carga lineal uniforme e infinitamente larga de densidad de carga $\lambda$ $\frac{\unit{C}}{\unit{m}}$.

El extremo más próximo de la barra a la carga lineal dista de esta la longitud $d$.

La barra posee una carga total $\mathrm{Q}$ distribuida uniformemente en toda su longitud.

Determinar la fuerza que la carga lineal ejerce sobre la barra.

\vspace{0.5cm}

\begin{center}
    \begin{tikzpicture}[scale=1.2]
        % Parameters
        \def\d{2.0} % Distance to bar
        \def\L{3.0} % Length of bar
        \def\w{0.3} % Width of bar drawing
        
        % Infinite Line Charge (x-axis)
        \draw[thick, blue] (-2.5, 0) -- (2.5, 0);
        \draw[dashed, blue] (2.5, 0) -- (3.0, 0);
        \draw[dashed, blue] (-3.0, 0) -- (-2.5, 0);
        
        % Positive charges on the line
        \foreach \x in {-2.2, -1.8, ..., 2.2} {
            \node[blue, font=\tiny] at (\x, 0) {$+$};
        }
        \node[blue, above] at (2, 0) {$\lambda$};

        % The Bar (on y-axis)
        \draw[fill=gray!20, thick] (-\w/2, \d) rectangle (\w/2, \d+\L);
        \node[right] at (\w/2, \d+\L/2) {$\mathrm{Q}$};
        
        % Dimension lines
        \draw[<->] (-1.0, 0) -- (-1.0, \d) node[midway, left] {$d$};
        \draw[<->] (-1.0, \d) -- (-1.0, \d+\L) node[midway, left] {$L$};
        
        % Helper lines for dimensions
        \draw[dotted] (-\w/2, \d) -- (-1.2, \d);
        \draw[dotted] (-\w/2, \d+\L) -- (-1.2, \d+\L);
        \draw[dotted] (-2.5, 0) -- (-1.0, 0);

    \end{tikzpicture}
\end{center}
\end{tcolorbox}

\vspace{0.5cm}

\begin{tcolorbox}[
    enhanced,
    frame hidden,
    colback=gray!15,
    arc=0pt,
    outer arc=0pt,
    left=0.5cm,
    overlay={
        \node[rotate=90, anchor=south, font=\fontsize{8}{10}\selectfont\itshape, opacity=0.8] at ([xshift=-0.2cm]frame.west) {Planteamiento};
    }
]

La ley de Gauss relaciona el flujo eléctrico que atraviesa una superficie cualquiera cerrada $S$ con la carga encerrada por la misma:

\begin{equation*}
    \oint_S \vec{E} \cdot d\vec{S} = \frac{\mathrm{Q}_{\text{enc}}}{\varepsilon_0}
\end{equation*}

Por otro lado, la fuerza se relaciona con el campo eléctrico como:

\begin{equation*}
    d\vec{F} = \vec{E} \, dq
\end{equation*}

\end{tcolorbox}


\section*{Solución}

Primero se calcula el campo eléctrico generado por el hilo infinito de la siguiente manera:

Como la línea cargada se encuentra sobre el eje $X$, por simetría el campo eléctrico será radial y dependerá únicamente de la distancia a la línea. Aplicando la Ley de Gauss a una superficie cilíndrica de radio $r$ y longitud $L'$ coaxal con la línea:

\begin{align*}
    \oint_S \vec{E} \cdot d\vec{S} &= \int_{\text{lat}} \vec{E} \cdot d\vec{S} + \int_{\text{tapas}} \vec{E} \cdot d\vec{S} = E (2\pi r L') + 0 \\[10pt]
    \mathrm{Q}_{\text{enc}} &= \lambda L'
\end{align*}

Igualamos:

\begin{equation*}
    E (2\pi r L') = \frac{\lambda L'}{\varepsilon_0} \implies E = \frac{\lambda}{2\pi \varepsilon_0 r}
\end{equation*}

Vectorialmente:

\begin{equation*}
    \vec{E} = \frac{\lambda}{2\pi \varepsilon_0 y} \, \vec{j} \quad \left[\frac{\unit{N}}{\unit{C}}\right]
\end{equation*}

Por otro lado, calculamos la fuerza como:

\begin{equation*}
    \vec{F} = \int d\vec{F}
\end{equation*}

Donde:

\begin{equation*}
    d\vec{F} = \vec{E} \, dq
\end{equation*}

Y:

\begin{tcolorbox}[
    enhanced,
    frame hidden,
    interior hidden,
    sidebyside,
    sidebyside align=center,
    lefthand width=0.7\textwidth,
    segmentation style={red!50, dashed, thick}
]

\begin{equation*}
    dq = \colorbox{orange!20}{$\displaystyle \frac{\mathrm{Q}}{L}$} \, dy
\end{equation*}

\tcblower

\noindent distribución lineal del segmento vertical finito

\end{tcolorbox}



\end{document}
